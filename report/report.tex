\documentclass[12pt]{article}

\usepackage{amsmath}
\usepackage{amsthm}

\usepackage{titlesec}

\usepackage{graphicx}
\graphicspath{{./Pictures/}}

\usepackage{xcolor-solarized}
\usepackage{pagecolor}
\color{solarized-base02}

\usepackage[T1]{fontenc}
\usepackage[utf8]{inputenc}

\usepackage[a4paper]{geometry}
\geometry{
    inner=20mm,
    outer=20mm,
    top=30mm,
    bottom=30mm,
    heightrounded,
    marginparwidth=50pt,
    marginparsep=20pt,
    headsep=25pt,
    headheight=30pt
}

\usepackage{hyperref}
\hypersetup{
    colorlinks=true,
    linkcolor=solarized-green,
    urlcolor=solarized-orange,
	pdftitle={Adaptive 1D 1st-order Lagrange FEM},
    pdfpagemode=FullScreen,
	pdfauthor={Andrea Di Antonio}
}

\usepackage{fancyhdr}
\pagestyle{fancy}
\fancyhf{}
\fancyhead[R]{Andrea Di Antonio}
\fancyhead[L]{Adaptive 1D 1st-order Lagrange FEM}
\fancyfoot[C]{Page \thepage}

\title{Adaptive 1D 1st-order Lagrange FEM}
\author{Andrea Di Antonio, 858798 \\ \hyperlink{mailto:a.diantonio1@campus.unimib.it}{a.diantonio1@campus.unimib.it}}
\date{July 27, 2023}

\begin{document}
	\maketitle
	\thispagestyle{fancy}
	
	\begin{abstract}
		\begin{center}
            Report for the course \textit{Metodi Numerici per Equazioni alle Derivate Parziali} on the definition and costruction of an \textit{Adaptive 1D 1st-order Lagrange FEM}\footnote{Written in MATLAB.} and its subsequent analysis.
        \end{center}
	\end{abstract}

    \tableofcontents
    
    \newpage
    \section{Problem Introduction}
    \subsection{Poisson Problem}

Let's start by considering the following problem, given $\Omega = (0, 1)$:
\begin{gather}
	\begin{cases}
		-u^{\prime \prime}(x) = f(x) \quad \text{in } \Omega \\
		u(0) = u(1) = 0
	\end{cases},
\end{gather}
which is the \textit{Poisson equation} on $\Omega$ with \textit{Dirichlet's boundary conditions}.

We'll consider a source $f(x)$ evaluated by considering the exact solution $u_{\alpha}$:
\begin{gather}
	u_{\alpha}(x) = x^{\alpha} (1 - x),
\end{gather}
on $\alpha > 1.5$, so that:
\begin{gather}
	f(x) = \alpha (\alpha + 1) x^{\alpha - 1} - \alpha (\alpha - 1) x^{\alpha - 2}.
\end{gather}

\subsubsection{Solution's Sobolev Regularity}

% To be discussed

\subsection{Weak Formulation and Finite Element Discretization} \label{fem_definition}

The weak formulation for the problem requires to find a $u \in V = H_0^1(\Omega)$ such that:
\begin{gather}
	a(u, v) = (f, v)_{0, \Omega} \quad \forall v \in V,
\end{gather}
where:
\begin{gather}
	a(u, v) = (u^{\prime}, v^{\prime})_{0, \Omega}.
\end{gather}

By introducing $V_h$ as the linear Lagrangian finite element space on a given partition $T_h$ of $\Omega$ we have the finite element discretization which requires to find a $u_h \in V_h$ such that:
\begin{gather}
	a(u_h, v_h) = (f, v_h)_{0, \Omega} \quad \forall v_h \in V_h,
\end{gather}

    \newpage
    \section{Finite Element Method}

    \newpage
    \section{Adaptive Method}
\end{document}