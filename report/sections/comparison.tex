\subsection[Error Comparison]{$|u - u_h|_{1, \Omega}$ Error Comparison}

Let's now compare the error $|u - u_h|_{1, \Omega}$ between a sequence of uniformly refined meshes and the adaptive method we just defined.

Since there is no mesh size concept for the adaptive refined meshes\footnote{The adaptively refined mesh fails to meet the definition of a quasi-uniform mesh.}, we will compare the errors against the number of elements.

\begin{figure}[!ht]
	\centering
	\includegraphics[width=15cm]{comparison.pdf}
	\caption{Comparison between errors against number of elements.}
\end{figure}

As we can see the adaptive method works for both values of $\alpha$, showing great results for $\alpha = 10$ and more or less the same behaviour as the standard method for $\alpha = 5/3$.

\newpage
\begin{multicols}{2}
	\lstinputlisting{../results/comparison.txt}
\end{multicols}

\newpage
\subsection{Stiffness Matrix Conditioning Comparison}

Given what we said on \nameref{condition}:
\begin{gather}
	\chi(\underline{\underline{A}}) \approx h^{-2},
\end{gather}
we may have some problems for the condition number of the stiffness matrix in the adaptive algorithm as the mesh size can get very small for some elements even with a low number of total nodes.

\begin{figure}[!ht]
	\centering
	\includegraphics[width=15cm]{comparisonCond.pdf}
	\caption{Comparison between stiffness matrix conditioning against number of elements.}
\end{figure}

We can observe that when the number of elements is fixed, the condition number is higher for the adaptive method matrix, even though it does stay below the values reached by the uniform method thus granting the the relative irrelevance of numerical errors.

\newpage
\begin{multicols}{2}
	\lstinputlisting{../results/comparisonCond.txt}
\end{multicols}