\subsection{Error Estimation and Refinement}

Instead of considering a sequence of uniformly refined meshes, we want to consider a new algorithm which locally\footnote{On every element of the mesh.} evaluates an error estimator and marks a certain number of elements which will be then refined.

We consider the following local error estimator:
\begin{gather} \label{estimator}
	\eta_K^2 = h_K \lVert f + \Delta u_h \rVert_{0, K}^2 + \frac{1}{2} \sum_{e \in E_K} h_K \lVert \llbracket \underline{\nabla} u_h \rrbracket \rVert_{0, e}.
\end{gather}

In our 1D case we can rewrite \ref{estimator} as it follows\footnote{$K = [x_{n - 1}, x_n]$.}:
\begin{gather}
	\eta_K^2 = h_K \lVert f \rVert_{0, K}^2 + \frac{1}{2} h_K \left[ \lvert \llbracket u_h^\prime \rrbracket (x_{n - 1}) \rvert + \lvert \llbracket u_h^\prime \rrbracket (x_n) \rvert \right].
\end{gather}

By evaluating $\eta_K$ $\forall K \in \Tau_h$, we can then consider an arbitrary number of elements with the highest local error, say the 25\%, and mark these elements. The last step of this algorithm would be, considering all the marked elements, to refine these by adding a new node at their midpoint.

The algorithm should be summarized in the following way:
\begin{description}
	\item[Solve] Solve the FEM problem on the \textit{starting mesh}.
	\item[Estimate] Estimate the local error on every element of the mesh.
	\item[Mark] Mark the element with the highest local error.
	\item[Refine] Refine the marked elements.  
	\item[Repeat] Repeat the algorithm an arbitrary number of times. 
\end{description}