\subsection{Linear Lagrangian Finite Element Space Base}

Given $\Omega = (0, 1)$ and $\Tau_h$ a partition on $\Omega$, we'd have to define a base on $V_h$ which is, as discussed before, the linear Lagrangian finite element space on $\Tau_h$.

Let $x_j$ a node for $\Tau_h$, we define $\phi_j$ as it follows:
\begin{gather}
	\phi_j(x_i) = \delta_{ij}.
\end{gather}

Moreover, the functions $\phi_j(x)$ are continuos and piecewise linear, which satisfies $\phi_j(x) \in V_h$ $\forall j \in \{1, \dots, N_{V_h}\}$.

\subsection{Finite Element Discretization}

As seen on \nameref{fem_definition}, the finite element discretization consists in finding $u_h \in V_h$ such that:
\begin{gather}
	a(u_h, v_h) = (f, v_h)_{0, \Omega} \quad \forall v_h \in V_h.
\end{gather}

Consider $\{\phi_j\}_j$ base on $V_h$ so that we can write:
\begin{gather}
	u_h(x) = \sum_{j = 1}^{N_{V_h}} u_j \phi_j(x),
\end{gather}
where $N_{V_h} = \dim(V_h)$ and $\underline{u} \in \R^{N_{V_h}}$, so that it is sufficient considering:
\begin{gather}
	a(u_h, \phi_j) = (f, \phi_j)_{0, \Omega} \quad \forall j \in \{1, \dots, N_{V_h}\}.
\end{gather}

This lets us ultimately write the following linear system:
\begin{gather}
	\underline{\underline{A}} \underline{u} = \underline{f},
\end{gather}
where $A_{ij} = a(\phi_i, \phi_j)$, $f_i = (f, \phi_i)_{0, \Omega}$.