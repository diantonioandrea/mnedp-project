\subsection{Poisson Problem}

Let's start by considering the following problem, given $\Omega = (0, 1)$:
\begin{gather}
	\begin{cases}
		-u^{\prime \prime}(x) = f(x) \quad \text{in } \Omega \\
		u(0) = u(1) = 0
	\end{cases},
\end{gather}
which is the \textit{Poisson equation} on $\Omega$ with \textit{Dirichlet's boundary conditions}.

We'll consider a source $f(x)$ evaluated by considering the exact solution $u_{\alpha}$:
\begin{gather}
	u_{\alpha}(x) = x^{\alpha} (1 - x),
\end{gather}
on $\alpha > 1.5$, so that:
\begin{gather}
	f(x) = \alpha (\alpha + 1) x^{\alpha - 1} - \alpha (\alpha - 1) x^{\alpha - 2}.
\end{gather}

Through the analysis of this problem, we'll stick to these two cases: $\alpha = 5/3, 10$.

\subsubsection{Solution's Sobolev Regularity} \label{sob_regularity}

Let's discuss the Sobolev regularity for the two cases: $\alpha = 5/3, 10$.

Given $\alpha = 10$, we have that $\forall k \in \N$:
\begin{gather}
	u_{10}(x) = x^{10} (1 - x) \in H^{k}(\Omega),
\end{gather}
being $u_{10}(x)$ a polynomial function, so that we can conclude that $u_{10}$ has a Sobolev regularity of $+\infty$.

Given $\alpha = 5/3$, however, we have that:
\begin{gather}
	u_{5/3}(x) = x^{5/3} - x^{7/3} \in H^{k}(\Omega),
\end{gather}
$\forall k \in \{0, 1, 2\}$, which are the only values of $k$ for which $D^{\gamma} u_{5/3}(x) \in L^2(\Omega)$ given $\gamma \le k$, so that we can conclude that $u_{10}$ has a Sobolev regularity of $2$.

\subsection{Weak Formulation and Finite Element Discretization} \label{fem_definition}

The weak formulation for the problem requires to find a $u \in V = H_0^1(\Omega)$ such that:
\begin{gather}
	a(u, v) = (f, v)_{0, \Omega} \quad \forall v \in V,
\end{gather}
where:
\begin{gather}
	a(u, v) = (u^{\prime}, v^{\prime})_{0, \Omega}.
\end{gather}

By introducing $V_h$ as the linear Lagrangian finite element space on a given partition $T_h$ of $\Omega$ we have the finite element discretization which requires to find a $u_h \in V_h$ such that:
\begin{gather}
	a(u_h, v_h) = (f, v_h)_{0, \Omega} \quad \forall v_h \in V_h.
\end{gather}